\section{Justificación}
\subsection{Justificación práctica}
	
A nivel mundial, tener conocimiento del tiempo, el clima e información sobre el
agua está aumentando, debido a la necesidad de los gobiernos de minimizar las
crecientes pérdidas económicas y ayudar a los países a adaptarse al cambio
climático. Estos factores afectan a las sociedades y economías a través de 
eventos extremos, tales como inundaciones, heladas, friajes, sequías 
prolongadas, y eventos climáticos de alto impacto que afectan la demanda de 
electricidad y la capacidad de producción, fechas de siembra y cosecha, gestión 
de la construcción, redes de transporte e inventarios, y salud humana entre los 
temas más importantes.
	
\subsection{Justificación académica}

El estudio de secuencias espaciotemporales con métodos de aprendizaje profundo 
está aún en sus primeras etapas en contraste con los tradicionales métodos 
numéricos. Estos métodos numéricos realizan simulaciones físicas para estimar
parámetros climáticos, estas simulaciones requieren un gran poder computacional.
A diferencia de los métodos numéricos, el uso de modelos de aprendizaje 
profundo tiene menores requerimientos computacionales. Los modelos de 
aprendizaje profundo se benefician de la gran cantidad de datos disponibles.





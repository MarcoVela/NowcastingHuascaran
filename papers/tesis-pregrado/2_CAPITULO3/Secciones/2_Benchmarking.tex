\section{Benchmarking}

%TODO: Corregir pesos de factores y validar
Se realizará una comparación entre los métodos descritos en el estado del arte. 
Se utilizarán criterios con pesos asignados siguiendo el Proceso Analítico 
Jerarquico (Analytical Hierarchical Process, AHP).
		
\subsection{Criterios}

\begin{itemize}
  \item \textbf{C1. Complejidad de entrenamiento:} Es el tiempo que 
  el modelo necesita para generalizar los datos, estimado en base a la 
  complejidad del modelo por usando el número de parámetros como indicador.
  \item \textbf{C2. Desempeño a muy corto plazo:} Es el desempeño definido por 
  las métricas de evaluación pertinentes.
  \item \textbf{C3. Explicabilidad:} Es la capacidad de reconocer los factores 
  que determinan el resultado del modelo. Es importante para poder obtener un 
  mayor conocimiento del fenómeno de estudio.
  \item \textbf{C4. Replicabilidad:} Es la disponibilidad de código fuente 
  y datasets que soporten el desarrollo de un tipo de método.
\end{itemize}

\subsection{Cálculo de pesos}
Se utilizó la escala de Saaty (ver tabla \ref{tab:escala_saaty}), para poder 
determinar la importancia relativa de los criterios respecto a ellos mismos.

\begin{table}[H]
  \centering
  \caption[Escala de Saaty]{Escala de Saaty.}
  \begin{tabular}{l|l}
  \textbf{Intensidad} & \textbf{Definición}    \\ \hline
  1                   & Igual importancia      \\
  3                   & Moderada importancia   \\
  5                   & Importancia fuerte     \\
  7                   & Importancia muy fuerte \\
  9                   & Importancia extrema    \\
  2, 4, 6, 7          & Valores intermedios   
  \end{tabular}
  \label{tab:escala_saaty}
\end{table}


\begin{table}[H]
  \centering
  \caption{Matriz de comparaciones pareadas.}
  \begin{tabular}{l|cccc|c|c}
     & C1 & C2  & C3  & C4  & \textbf{Suma} & \textbf{Peso}  \\ \hline
  C1 & 1  & 1/3 & 1/5 & 1/8 &  1.658 & 0.057\\
  C2 & 3  & 1   & 1/2 & 1/3 &  4.83  & 0.167\\
  C3 & 5  & 2   & 1   & 1/2 &  8.5   & 0.293\\
  C4 & 8  & 3   & 2   & 1   &  14    & 0.483\\ \hline
  \textbf{Suma} & 17     & 6.333       & 3.7       & 1.958 & 28.99  & 1.00   \\
  \end{tabular}
  \label{tab:comp_pareadas}
\end{table}

Se calculará el indice de consistencia ($CI$) de la siguiente forma:

$$CI = \frac{\lambda_{max}-n}{n-1}$$

para $n = 4$ nuestra asignación de pesos será consistente si $\frac{CI}{0.9} < 0.1$.

De la tabla \ref{tab:comp_pareadas} obtenemos $\lambda_{max} = 4.0587$, y 
$CI = 0.0196$, tenemos entonces que $\frac{CI}{0.9} \approx 0.022 < 0.1$ y 
nuestra asignación es consistente.

En la tabla \ref{tab:comp_pareadas} obtenemos que el criterio 4 
(Replicabilidad) es el más importante, seguido de Explicabilidad, Desempeño a 
corto plazo y finalmente complejidad de entrenamiento. Esto se dió debido a 
que el stakeholder valora más una solución funcional que pueda ir mejorando con 
el tiempo a una solución compleja cuya implementación y explicabilidad sea 
baja.

\subsection{Evaluación}
Las alternativas a evaluar son:

\begin{itemize}
  \item \textbf{A1. DGM}
  \item \textbf{A2. 2D \& 3D CNN} 
  \item \textbf{A3. ConvLSTM \& TrajGRU} 
  \item \textbf{A4. Atention Based}
\end{itemize}

Para la evaluación entre alternativas usaremos una escala de 0 a 5 donde 0 
indica la peor opción para el criterio y 5 la mejor.

\begin{table}[H]
  \centering
  \caption{Evaluación de las alternativas.}
  \begin{tabular}{l|l|llll}
        & Pesos & A1   & A2   & A3   & A4   \\ \hline
  C1    & 0.057 & 1    & 4    & 4    & 2    \\
  C2    & 0.167 & 4    & 1    & 3    & 4    \\
  C3    & 0.293 & 2    & 4    & 4    & 2    \\
  C4    & 0.483 & 0    & 3    & 3    & 0    \\ \hline
  Total & 1     & 1.31 & 3.02 & 3.35 & 1.37
  \end{tabular}
  \label{tab:evaluacion}

\end{table}

\subsection{Decisión}

De acuerdo a los puntajes obtenidos en la tabla \ref{tab:evaluacion}, se elige 
la alternativa 3 (ConvLSTM \& TrajGRU).

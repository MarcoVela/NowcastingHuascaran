\section{PRIMERA SECCIÓN}

En la figura anterior se observa que:
\begin{subequations}\label{Cap3_Eq1}
\begin{gather}
u_{iT}=u_{g}+u_{b}+u_{i} \hspace{5mm} \rightarrow \hspace{5mm} \ddot{u}_{iT}=\ddot{u}_{g}+\ddot{u}_{b}+\ddot{u}_{i}		\label{Cap3_Eq1_1} \\[2 mm]
u_{bT}=u_{g}+u_{b}\hspace{5mm} \rightarrow \hspace{5mm} \ddot{u}_{bT}=\ddot{u}_{g}+\ddot{u}_{b} \label{Cap3_Eq1_2}
\end{gather}
\end{subequations}
Remplazando y simplificando se puede llegar a una expresión matricial de la forma: 
\begin{equation}\label{Cap3_Eq3}
\mathbfit{M_{s}\ddot{u}_{s}}(t)+\mathbfit{M_{s}\tau}\ddot{u}_{b}(t)+\mathbfit{C_{s}\dot{u}_{s}}(t)+\mathbfit{K_{s}u_{s}}(t)=-\mathbfit{M_{s}\tau}\ddot{u}_{g}(t)
\end{equation}

\begin{equation}\label{Cap3_Eq7}
\mathbfit{\overline{M}\ddot{u}}(t)+\mathbfit{\overline{C}\dot{u}}(t)+\mathbfit{\overline{K}u}(t)+\mathbfit{\overline{R}}(t)=-\mathbfit{v}\ddot{u}_{g}(t)
\end{equation}
Donde: 
\begin{subequations}\label{Cap3_Eq8}
\begin{gather}
\mathbfit{\overline{M}}=\begin{bmatrix}
\mathbfit{M_{s}} & \mathbfit{M_{s} \tau} \\ 
\mathbfit{\tau}^{T} \mathbfit{M_{s}}  & m_{tot}
\end{bmatrix};\hspace{3mm} 
\mathbfit{\overline{C}}=\begin{bmatrix}
\mathbfit{C_{s}}  & \mathbf{0}\\ 
\mathbf{0}^{T}      & 0
\end{bmatrix}; \hspace{3mm} 
\mathbfit{\overline{K}}=\begin{bmatrix}
\mathbfit{K_{s}}  & \mathbf{0}\\ 
\mathbf{0}^{T}      & 0
\end{bmatrix};		\label{Cap3_Eq8_1} \\[2 mm]
\mathbfit{\overline{R}}(t)=\begin{bmatrix}
\mathbf{0}\\ 
R(t)
\end{bmatrix};\hspace{6mm} 
\mathbfit{u}(t)=\begin{bmatrix}
\mathbfit{u_{s}}(t)\\ 
u_{b}(t)
\end{bmatrix};\hspace{3mm} 
\mathbfit{v}=\begin{bmatrix}
\mathbfit{M_{s} \tau} \\ 
m_{tot}
\end{bmatrix}
		\label{Cap3_Eq8_2}
\end{gather}
\end{subequations}


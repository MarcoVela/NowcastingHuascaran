\section{Conjunto de datos}

El desarrollo del modelo basado en aprendizaje profundo requiere un conjunto de 
datos apropiado para el entrenamiento. Para poder realizar el pronóstico de 
tormentas eléctricas se utilizarán los datos de densidad de destellos por área 
(Flash Extend Density, FED), obtenida a partir del producto GLM-L2-LCFA 
(Geostationary Lightning Mapper Level 2 Lightning Detection) del satélite 
GOES-16.

\subsection{Geostationary Lightning Mapper Level 2 Lightning Detection}
El producto de detección de rayos contiene una lista de destellos de rayo, 
acompañado de sus grupos y eventos constituyentes.

\textbf{Evento: }Un evento es definido como la ocurrencia de un solo pixel de sensor que excede 
el umbral base durante un periodo de 2ms. Se debe señalar que un evento puede 
ser ocasionado por ruido que exceda el umbral base, en ese caso el evento es 
una falsa alarma.

\textbf{Grupo: }La descarga de un rayo normalmente ilumina más de un pixel durante el intervalo 
de tiempo. El resultado es dos o más eventos adyacentes durante el mismo 
periodo de tiempo. Cuando estos eventos múltiples sean adyacentes (compartan 
un lado o esquina en común), serán colocados en el mismo grupo. La definición 
formal de un grupo es uno o más eventos simultaneos registrados en pixeles 
adyacentes.

\textbf{Destello: }Un destello de rayo es definido como un conjunto de grupos 
secuencialmente separados en el tiempo por 330ms o menos y en espacio por no 
más de 16.5km. Para que dos grupos sean considerados parte del mismo destello, 
es necesario que dos eventos cualquiera de los grupos cumplan la separación de 
330ms y 16.5km. No se utilizan los centroides de los grupos para determinar 
si dos grupos son parte de un destello.

\begin{figure}[H]
  \centering
  \includegraphics[width=6cm]{E_IMAGENES/5_Metodologia/flash_group_event}
  \caption[Destello del GLM]{
    Ilustración de un destello del GLM compuesto de 2 grupos y 20 eventos.
    \newline
    Fuente: \citep{GOODMAN201334}
  }
  \label{fig:fed}
\end{figure}

\subsubsection{Características de los datos publicados}
Los datos son distribuidos por la National Oceanic and Atmospheric 
Administration (NOAA) en archivos NetCDF-v4 a través de Amazon Web Services 
(AWS) o Google Cloud Platform (GCP) con frecuencia de 20 segundos. Cada archivo 
contiene:

\textbf{Dimensiones: }
\begin{itemize}
  \item number\_of\_flashes
  \item number\_of\_groups
  \item number\_of\_events
  \item number\_of\_time\_bounds
  \item number\_of\_field\_of\_view\_bounds
  \item number\_of\_wavelength\_bounds
\end{itemize}

\begin{table}[H]
  \centering
  \small
  \label{tab:vars_events_glm}
  \caption{
    Variables de eventos contenidas en un archivo del producto GLM-L2-LCFA del 
    satélite GOES-16
  }
  \begin{tabular}{l|p{3cm}|p{1.15cm}|p{3.5cm}}
    \textbf{Nombre} & 
    \textbf{Nombre largo} & 
    \textbf{Tipo de dato} & 
    \textbf{Atributos} \\ \hline

    event\_id & 
    Identificador único para el producto de eventos de rayo. &
    Int32 &
    {Sin signo}\\ \hline

    event\_time\_offset &
    Tiempo de ocurrencia del evento. &
    Int16 &
    \parbox[t]{3.5cm}{Sin signo \\ Escalado \\ Compensado \\ Medido en segundos desde una fecha}\\ \hline
    
    event\_lat &
    Coordenada de latitud del evento. &
    Int16 &
    \parbox[t]{3.5cm}{Sin signo \\ Escalado \\ Compensado \\ Medido en grados norte}\\ \hline

    event\_lon &
    Coordenada de longitud del evento. &
    Int16 &
    \parbox[t]{3.5cm}{Sin signo \\ Escalado \\ Compensado \\ Medido en grados este}\\ \hline

    event\_energy &
    Energía radiante del evento &
    Int16 &
    \parbox[t]{3.5cm}{Sin signo \\ Escalado \\ Compensado \\ Medido en Joules}\\ \hline

    event\_parent\_group\_id &
    Identificador único para el producto del grupo al que pertenece el evento. &
    Int32 &
    \parbox[t]{3.5cm}{Sin signo}\\
  \end{tabular}
\end{table}

\begin{table}[H]
  \centering
  \small
  \label{tab:vars_groups_glm}
  \caption{
    Variables de grupos contenidas en un archivo del producto GLM-L2-LCFA del 
    satélite GOES-16
  }
  \begin{tabular}{l|p{3cm}|p{1.15cm}|p{3.5cm}}
    \textbf{Nombre} & 
    \textbf{Nombre largo} & 
    \textbf{Tipo de dato} & 
    \textbf{Atributos} \\ \hline

    group\_id &
    Identificador único para el producto del grupo.&
    Int32 &
    \parbox[t]{3.5cm}{Sin signo}\\ \hline

    group\_time\_offset &
    Tiempo de ocurrencia promedio de los eventos constituyentes del grupo.&
    Int16 &
    \parbox[t]{3.5cm}{Sin signo \\ Escalado \\ Compensado \\ Medido en segundos desde una fecha}\\ \hline

    group\_frame\_time\_offset &
    Tiempo de ocurrencia promedio de los eventos constituyentes del grupo.&
    Int16 &
    \parbox[t]{3.5cm}{Sin signo \\ Escalado \\ Compensado \\ Medido en segundos desde una fecha}\\ \hline

    group\_lat &
    Centroide del grupo (media ponderada de los eventos por su energía).&
    Float32 &
    \parbox[t]{3.5cm}{Medido en grados norte}\\ \hline

    group\_lon &
    Centroide del grupo (media ponderada de los eventos por su energía).&
    Float32 &
    \parbox[t]{3.5cm}{Medido en grados este}\\ \hline

    group\_area &
    Cobertura de área por grupo (pixeles que contienen al menos un evento constituyente).&
    Int16 &
    \parbox[t]{3.5cm}{Sin signo \\ Acotado \\ Escalado \\ Compensado \\ Medido en m$^2$}\\ \hline

    group\_energy &
    Energía radiante del grupo&
    Int16 &
    \parbox[t]{3.5cm}{Sin signo \\ Acotado \\ Escalado \\ Compensado \\ Medido en Joules}\\ \hline

    group\_parent\_flash\_id &
    Identificador unico para el producto del destello asociado al grupo.&
    Int16 &
    \parbox[t]{3.5cm}{Sin signo}\\ \hline

    group\_quality\_flag &
    Indicador de calidad de los datos del grupo.&
    Int16 &
    \parbox[t]{3.5cm}{Sin signo \\ Acotado}\\ 

  \end{tabular}
\end{table}

\begin{table}[H]
  \centering
  \small
  \label{tab:vars_flash_glm}
  \caption{
    Variables de destellos contenidas en un archivo del producto GLM-L2-LCFA del 
    satélite GOES-16
  }
  \begin{tabular}{l|p{3cm}|p{1.15cm}|p{3.5cm}}
    \textbf{Nombre} & 
    \textbf{Nombre largo} & 
    \textbf{Tipo de dato} & 
    \textbf{Atributos} \\ \hline

    group\_id &
    Identificador único para el producto del destello.&
    Int16 &
    \parbox[t]{3.5cm}{Sin signo}\\ \hline

    flash\_time\_offset\_of\_first\_event &
    Tiempo de ocurrencia del primer evento constituyente del destello. &
    Int16 &
    \parbox[t]{3.5cm}{Sin signo \\ Escalado \\ Compensado \\ Medido en segundos desde una fecha}\\ \hline

    flash\_time\_offset\_of\_last\_event &
    Tiempo de ocurrencia del último evento constituyente del destello. &
    Int16 &
    \parbox[t]{3.5cm}{Sin signo \\ Escalado \\ Compensado \\ Medido en segundos desde una fecha}\\ \hline

    flash\_frame\_time\_offset\_of\_first\_event &
    Tiempo de ocurrencia del primer evento constituyente del destello. &
    Int16 &
    \parbox[t]{3.5cm}{Sin signo \\ Escalado \\ Compensado \\ Medido en segundos desde una fecha}\\ \hline

    flash\_frame\_time\_offset\_of\_last\_event &
    Tiempo de ocurrencia del último evento constituyente del destello. &
    Int16 &
    \parbox[t]{3.5cm}{Sin signo \\ Escalado \\ Compensado \\ Medido en segundos desde una fecha}\\ \hline

    flash\_lat &
    Centroide del destello (media ponderada de los eventos por su energía). &
    Float32 &
    \parbox[t]{3.5cm}{Medido en grados norte}\\ \hline

    flash\_lon &
    Centroide del destello (media ponderada de los eventos por su energía). &
    Float32 &
    \parbox[t]{3.5cm}{Medido en grados este}\\ \hline

    flash\_area &
    Cobertura de área por destello (pixeles que contienen al menos un evento constituyente). &
    Int16 &
    \parbox[t]{3.5cm}{Sin signo \\ Acotado \\ Escalado \\ Compensado \\ Medido en m$^2$}\\ \hline

    flash\_energy &
    Energía radiante del destello. &
    Int16 &
    \parbox[t]{3.5cm}{Sin signo \\ Acotado \\ Escalado \\ Compensado \\ Medido en Joules}\\ \hline

    flash\_quality\_flag &
    Indicador de calidad de los datos del destello. &
    Int16 &
    \parbox[t]{3.5cm}{Sin signo \\ Acotado }\\ 

  \end{tabular}
\end{table}

\subsection{Flash Extend Density}
Para poder utilizar el método de aprendizaje profundo escogido es necesario 
contar con una representación matricial de los datos. Esta transformación 
permite capturar la naturaleza espacial de los datos.

Al realizar el grillado de los datos se obtendrán archivos con las siguientes 
características:

\begin{itemize}
  \item Como mínimo 2 dimensiones de latitud y longitud.
  \item Una variable de densidad de destello cuyas dimensión sea 
  latitud\times longitud.
  \item Una variable que indique el tiempo inicial y final de los eventos 
  considerados.
\end{itemize}

Debe considerarse también la resolución espacial apropiada, debido a que otros 
productos del satélite GOES-16 cuentan con una resolución de 2km\times 2km 
se realizarán pruebas con múltiplos de dicha resolución.

\begin{figure}[H]
  \centering
  \includegraphics[width=8cm]{E_IMAGENES/5_Metodologia/fed_puntual}
  \caption{Densidad de Destellos por Área}{\begin{center}
    Densidad de Destellos por Área\newline
    Fuente: Elaboración propia
    \end{center}
  }
  \label{fig:fed_1km}
\end{figure}
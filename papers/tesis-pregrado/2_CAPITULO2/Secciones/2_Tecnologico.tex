\section{Marco Tecnológico}

\subsection{Aprendizaje automático}

Según \cite{mitchell1997machine}, el aprendizaje automático (Machine Learning) 
es el estudio de algoritmos que son capaces de mejorar su desempeño en tareas 
específicas por medio del uso de datos.

\subsection{Redes Neuronales Artificiales}

Las redes neuronales artificiales son un modelo compuesto por múltiples capas 
interconectadas de manera similar a las conexiones de neuronas en el cerebro 
humano. Las diferentes arquitecturas de redes neuronales les dan el potencial 
de ser aproximadores universales de funciones \cite{braspenning1995artificial}. 
% Quizá incluir imagenes del simil entre redes neuronales y el cerebro

\subsection{Aprendizaje profundo}

Las técnicas de aprendizaje automático adolecen de un problema fundamental. 
Requieren de dominio considerable del área de aplicación para extraer 
características de los datos y poder transformarlos en una representación 
adecuada para que el modelo pueda reconocer patrones. \cite{LeCun2015} definen
los métodos de aprendizaje profundo (Deep Learning) como métodos que permiten a 
las máquinas descubrir representaciones significativas en conjuntos de datos 
sin procesar.
% Incluir imagenes de un perceptron multicapas


\subsection{Redes Neuronales Convolucionales}

Las redes neuronales convolucionales (Convolutional Neural Networks, CNN) son 
una clase de red neuronal artificial inspiradas por procesos biológicos en los 
que las neuronas responden a estímulos en una región restringida del campo 
visual, denominado campo receptivo (Fukushima, \citeyear{Fukushima1980}). 
\cite{Ciresan2011FlexibleHP} popularizaron su uso al demostrar que la 
aceleración por hardware podía reducir el tiempo de procesamiento en un factor 
de 60. En ese mismo año, usaron dicha implementación para ganar un concurso de 
reconocimiento de imagenes, despertando más interés en la comunidad de visión 
por computador.
% Incluir imagenes de la operación de convolución
% Hablar sobre los parámetros


\subsection{Redes Neuronales Recurrentes}

Las redes neuronales recurrentes (Recurrent Neural Networks, RNN) son una clase 
de red neuronal artificial que permite modelar comportamiento secuencial a lo 
largo de una dimensión (usualmente temporal). Las RNN utilizan un estado interno 
denominado memoria para poder procesar una secuencia de longitud variable de 
entradas \cite{Abiodun2018}.
% Incluir imagenes sobre el tratamiento de secuencias
% Hablar sobre distintos tipos de RNN



\subsection{Métricas de evaluación}

MSE, F1, AUC, ETC...

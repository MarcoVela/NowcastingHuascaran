\section{Marco Legal y Normativo}

\subsection{Lineamientos para el diseño de sistemas integrados de vigilancia y 
pronóstico hidrometeorológico con fines de alerta temprana}

\cite{senamhi2021lineamientos} indica que los servicios y productos asociados a 
la alerta temprana deben obedecer el principio de subsidiariedad, lo que 
significa que los avisos de peligro a muy corto plazo deben poder ser generados 
por las oficinas desconcentradas del SENAMHI.\newline
La continuidad del servicio requiere además un sistema robusto y resiliente, 
siguiendo el principio de subsidiariedad, la sede central del SENAMHI apoyaría 
a los organos desconcentrados en la generación o entrega de productos en caso 
de que sus capacidades se vean sobrepasadas.

\subsection{Recommendation of the Council on Artificial Intelligence}

De acuerdo con \cite{oecd2019recom} uno de los principios para la 
administración responsable de la inteligencia artificial es la 
\textbf{Robustez, seguridad y protección}:
\begin{itemize}
  \item Los sistemas de IA deben ser robustos y seguros durante todo su ciclo 
  de   vida de tal manera que, en condiciones de uso normal, uso o maluso 
  previsible,   u otras condiciones adversas, funcionen apropiadamente y no 
  planteen un riesgo de seguridad no razonable.
  \item Para este fin, los actores involucrados deben asegurar la trazabilidad, 
  incluso en relación a los conjuntos de datos, procesos y decisiones hechas 
  durante el ciclo de vida del sistema, para habilitar el análisis de los 
  resultados del sistema.
  \item Los actores deberán, basado en sus roles, aplicar un enfoque 
  sistemático de gestión de riesgos a cada fase del ciclo de vida del sistema 
  para atender los riesgos relacionados a sistemas de IA, incluyendo seguridad 
  digital y sesgo.
\end{itemize}


% Lineamientos para el diseño de sistemas integrados de vigilancia y pronóstico hidrometeorológico con fines de alerta temprana
% ENIA Peru
